\documentclass[../main.tex]{subfiles}
\begin{document}
\section*{Opgave 6}
\subsection*{6a: Bestem længden af linjestykket DE}
    \subsubsection*{Metode}
    Vi bestemmer vektor \(\overrightarrow{DE}\) og beregner størrelsen af vektoren.
    \subsubsection*{Beregning}
    \(D \defeq \langle 0|0|2 \rangle\)\\
    \(E \defeq \langle 4|4|1.6 \rangle\)\\\\
    \(\overrightarrow{DE} \defeq E^{\%T}-D^{\%T} = \textcolor{blue}{\begin{pmatrix} 4\\ 4\\ -0.4 \end{pmatrix}}\)\\\\
    \(\left|\left| \overrightarrow{DE} \right|\right| = \textcolor{blue}{5.670978752}\)
    \subsubsection*{Konklusion}
    Længden af linjestykket DE er 5.67 enheder.
\clearpage
\subsection*{6b: Bestem en parameterfremstilling for den rette linje \textit{l}, der går gennem punkt \textit{A} og punkt \textit{C}}
    \subsubsection*{Metode}
        Vi ved at linjen \(l\) går igennem punkterne \(A\) og \(C\).\\
        For at bestemme en parameterfremstilling for linjen \(l\) bestemmer vi vektoren der går fra \(A\) til \(C\) og stedvektoren til \(A\).\\
        \[\begin{pmatrix} x\\y\\z \end{pmatrix}=\overrightarrow{A}+t\cdot \overrightarrow{AC}\]
        \(\overrightarrow{A}\) angiver placeringen i det tredimensionelle kartesiske koordanatsystem. Mens \(\overrightarrow{AC}\) angiver retningen og \(t\) sørger for at vi rammer alle punkter.
    \subsubsection*{Beregning}
        \(A \defeq \langle 0|0|0 \rangle\)\\
        \(C \defeq \langle 5|5|2 \rangle\)\\
        \(\overrightarrow{AC} \defeq C^{\%T}-A^{\%T} = \textcolor{blue}{\begin{pmatrix}5\\5\\2 \end{pmatrix}}\)\\
        \(\begin{pmatrix}x\\y\\z \end{pmatrix}=A^{\%T} + t\cdot \overrightarrow{AC}\)
        \[\textcolor{blue}{\begin{pmatrix}x\\y\\z \end{pmatrix} = t \cdot \begin{pmatrix}5\\5\\2 \end{pmatrix}}\]
    \subsubsection*{Konklusion}
        Parameterfremstillingen for den rette linje \(l\) kan skrives ved:
        \[\begin{pmatrix}x\\y\\z \end{pmatrix} = t \cdot \begin{pmatrix}5\\5\\2 \end{pmatrix}\]
\clearpage
\subsection*{6c: Bestem vinklen mellem linje \textit{m} og linje \textit{l}.}
    \subsubsection*{Metode}
        Først bestemmer vi en retningsvektor for linje \(l\) og \(m\).\\
        Derefter benytter vi sammenhængen med prikproduktet af to vektorer og vinklen imellem dem:
        \[\overrightarrow{a}\bullet \overrightarrow{b}=\left|\overrightarrow{a}\right|\cdot\left|\overrightarrow{b}\right|\cdot \text{cos}(v)\]
        Vi isolerer v.
    \subsubsection*{Beregning}
        Vektor \(\overrightarrow{DE}\) er en retningsvektor for linjen \(m\).\\
        Vektor \(\overrightarrow{AC}\) er en retningsvektor for linjen \(l\).\\\\
        \(\displaystyle v = solve\left(\overrightarrow{DE} \bullet \overrightarrow{AC} = \left|\left|\overrightarrow{DE} \right|\right| \cdot \left|\left|\overrightarrow{AC} \right|\right|\cdot \text{cos} (v)\right) \cdot \frac{360}{2\pi} = \textcolor{blue}{19.83786034}\)
    \subsubsection*{Konklusion}
        Vinklen mellem linje \(m\) og \(l\) er ca.  \(19.84\degree\).
\vspace*{1.5 cm}
\subsection*{6d: Bestem arealet af trekanten \textit{ABC}}
    \subsubsection*{Metode}
        Vi bestemmer arealet af trekanten \(ABC\) ved:
        \[A_\triangle=\frac{1}{2} \cdot \left|\left| \overrightarrow{AB} \times \overrightarrow{AC}\right|\right|\]
        Netop da det er gældende at \(\displaystyle \left|\overrightarrow{a} \times \overrightarrow{b} \right|=\left|\overrightarrow{a}\right|\cdot \left|\overrightarrow{b} \right|\cdot \text{sin}(v)\), det vi genkender som \(h\cdot g\). 
    \subsubsection*{Beregning}
        \(B \defeq \langle 3 | 0 | 0 \rangle = \textcolor{blue}{\begin{pmatrix} 3 & 0 & 0 \end{pmatrix}}\)\\\\
        \(\overrightarrow{AB} \defeq B^{\%T}-A^{\%T} = \textcolor{blue}{\begin{pmatrix} 3\\0\\0 \end{pmatrix}}\)\\\\
        \(\displaystyle A=\frac{1}{2}\cdot \left|\left| \overrightarrow{AB} \times \overrightarrow{AC}\right|\right| = \textcolor{blue}{8.077747210}\)
    \subsubsection*{Konklusion}
        Arealet af trekanten \(ABC\) er ca.  \(8.08 \;\text{enh}^2\).
\end{document}