\documentclass[../main.tex]{subfiles}
\begin{document}
\section*{Opgave 2}
\subsection*{2a: Bestem en parameterfremstilling for den rette linje \textit{l}.}
    \subsubsection*{Metode}
        Vi ved at linjen \(l\) går igennem punkterne \(A\) og \(D\).\\
        For at bestemme en parameterfremstilling for linjen \(l\) bestemmer jeg vektoren der går fra \(A\) til \(D\) og stedvektoren til \(A\).\\
        \[\begin{pmatrix} x\\y\\z \end{pmatrix}=\overrightarrow{A}+t\cdot \overrightarrow{AD}\]
        \(\overrightarrow{A}\) angiver placeringen i det tredimensionelle kartiesiske koordanatsystem. Mens \(\overrightarrow{AD}\) angiver retningen og \(t\) sørger for at vi rammer alle punkter.
    \subsubsection*{Beregning}
        \(A \defeq \langle 1{\vert 2\vert}0 \rangle\)\\
        \(B \defeq \langle 3{\vert 4\vert}0 \rangle\)\\
        \(C \defeq \langle 2{\vert 6\vert}0\rangle\)\\
        \(D \defeq \langle 2{\vert 5\vert}7 \rangle\)\\\\
        \(\overrightarrow{AD} \defeq D^{\%T} - A^{\%T}= \textcolor{blue}{\begin{pmatrix} 1\\ 3\\7 \end{pmatrix}}\)\\\\\\
        \(\langle x, y, z \rangle = A^{\%T}+ t\cdot \overrightarrow{AD}\)
        \textcolor{blue}{\[\begin{pmatrix} x \\ y\\ z\end{pmatrix}=\begin{pmatrix} 1+t \\ 2+3t\\ 7t\end{pmatrix}\]}
    \subsubsection*{Konklusion}
        Parameterfremstillingen for linjen \(l\) kan skrives ved:
        \[\begin{pmatrix} x \\ y\\ z\end{pmatrix}=\begin{pmatrix}1\\ 2\\0 \end{pmatrix}+ t \cdot \begin{pmatrix} 1 \\ 3\\ 7\end{pmatrix}\]
\clearpage
\subsection*{2b: Bestem en ligning for den plan \(\pi\), der indeholder punkterne \textit{A}, \textit{B} og \textit{D}.}
    \subsubsection*{Metode}
        Først bestemmes vektor \(\overrightarrow{AB}\), \(\overrightarrow{AD}\) bestemte vi i forrige opgave, vektorerne udspænder da et plan \(\pi\) der går gennem \(A,\, B,\text{ og } D\).\\
        Vi krydser \(\overrightarrow{AB}\) og \(\overrightarrow{AD}\) for at bestemme en normalvektor til planen.\\
        Dernest prikker vi normalvektoren med en vilkårlig vektor i planen.
        \[\overrightarrow{n}\bullet \left(\begin{pmatrix} x \\ y \\ z \end{pmatrix}-\begin{pmatrix}A_x \\A_y\\A_z \end{pmatrix}\right)=0\]
        Vi får planens ligning.
    \subsubsection*{Beregning}
        \(\overrightarrow{AB} \defeq B^{\%T} - A^{\%T}=\textcolor{blue}{\begin{pmatrix} 2 \\ 2 \\ 0 \end{pmatrix}}\)\\\\\\
        \(\overrightarrow{n} \defeq \overrightarrow{AB} \times \overrightarrow{AD} = \textcolor{blue}{\begin{pmatrix} 14 \\ -14 \\ 4 \end{pmatrix}}\)\\\\\\
        \(\overrightarrow{n}\bullet \left(\begin{pmatrix} x \\ y \\ z \end{pmatrix}-A^{\%T}\right)=0\)
        \textcolor{blue}{\[14x-14y+4z+14=0\]}
        \subsubsection*{Konklusion}
            Ligningen for planen \(\pi\) kan skrives ved:
            \[14x-14y+4z+14=0\]
\clearpage
\subsection*{2c: Bestem afstanden mellem punktet \textit{M} og linjen \textit{l}.}
    \subsubsection*{Metode}
        Vi ved at distancen mellem et punkt og en linje i rummet er givet ved:
        \[\text{dist}(P,l)=\frac{\left|\left| \overrightarrow{r} \times \overrightarrow{P_0 P}\right|\right|}{\left|\left|\overrightarrow{r} \right|\right|}\]
        hvor \(\overrightarrow{r}\) er en retningsvektor for linjen og \(\overrightarrow{P_0 P}\) er vektoren fra \(P_0\) til \(P\).
    \subsubsection*{Beregning}
        Da punktet \(M\) er midtpunkt på linjestykket \(CD\). 
        \begin{itemize}
            \item \(C\begin{pmatrix} 2 & 6 & 0 \end{pmatrix}\)
            \item \(D\begin{pmatrix} 2 & 5 & 7 \end{pmatrix}\)
        \end{itemize}
        Må punktet \(M\) have koordinaterne \(\begin{pmatrix} 2 & 5.5 & 3.5 \end{pmatrix}\).\\\\
        \(M \defeq \langle 2{\vert 5.5\vert}3.5 \rangle\)\\\\
        \(\overrightarrow{DM} \defeq M^{\%T}-D^{\%T} = \textcolor{blue}{\begin{pmatrix} 0 \\ 0.5 \\ -3.5 \end{pmatrix}}\)\\\\\\
        \(\displaystyle \text{dist}(M, l)= \frac{\left|\left| \overrightarrow{AD} \times \overrightarrow{DM}\right|\right|}{\left|\left|\overrightarrow{AD} \right|\right|}=\textcolor{blue}{1.88}\)\\\\
    \subsubsection*{Konklusion}
    Afstanden mellem punktet \textit{M} og linjen \textit{l} er 1.88.
\end{document}