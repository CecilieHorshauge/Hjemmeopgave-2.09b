\documentclass[../main.tex]{subfiles}
\begin{document}
\section*{Opgave 4}
\subsection*{4a: Bestem en parameterfremstilling for den rette linje \textit{l}}
    \subsubsection*{Metode}
        Vi ved at linjen \(l\) går igennem punkterne \(A\) og \(B\).\\
        For at bestemme en parameterfremstilling for linjen \(l\) bestemmer jeg vektoren der går fra \(A\) til \(B\) og stedvektoren til \(A\).\\
        \[\begin{pmatrix} x\\y\\z \end{pmatrix}=\overrightarrow{A}+t\cdot \overrightarrow{AB}\]
        \(\overrightarrow{A}\) angiver placeringen i det tredimensionelle kartesiske koordanatsystem. Mens \(\overrightarrow{AB}\) angiver retningen og \(t\) sørger for at vi rammer alle punkter.
    \subsubsection*{Beregning}
        \(A \defeq \langle 2|5|3 \rangle = \textcolor{blue}{\begin{pmatrix} 2 & 5 & 3 \end{pmatrix}}\)\\
        \(B \defeq \langle 3|7|-1 \rangle = \textcolor{blue}{\begin{pmatrix} 3 & 7 & -1 \end{pmatrix}}\)\\\\
        \(\overrightarrow{AB} \defeq B^{\%T} - A^{\%T}=\textcolor{blue}{\begin{pmatrix} 1 \\ 2 \\ -4 \end{pmatrix}} \)\\\\\\
        \(\begin{pmatrix} x \\ y \\ z \end{pmatrix}= A^{\%T}+t \cdot \overrightarrow{AB}\)
        \[\textcolor{blue}{\begin{pmatrix} x \\ y \\ z \end{pmatrix}=\begin{pmatrix} 2 \\ 5 \\ 3 \end{pmatrix} + t \cdot \begin{pmatrix} 1 \\ 2 \\ -4 \end{pmatrix}}\]
        \subsubsection*{Konklusion}
        Parameterfremstillingen for linjen \(l\) kan skrives ved:
        \[\begin{pmatrix} x \\ y \\ z \end{pmatrix}=\begin{pmatrix} 2 \\ 5 \\ 3 \end{pmatrix} + t \cdot \begin{pmatrix} 1 \\ 2 \\ -4 \end{pmatrix}\]
\clearpage
\subsection*{4b: Bestem en ligning for planen \(\pi\)}
    \subsubsection*{Metode}
        Vi har bestemt en retningsvektor for linjen \(l\), der også er en normalvektor for planen.\\
        Vi kender et fast punkt \(C=(3;\,4;\,8)\), der ligger i planen \(\pi\).\\
        For at finde planens ligning prikker vi normalvektoren \(\overrightarrow{n}\) med \(\overrightarrow{P_0P}\).
        \[\overrightarrow{n} \bullet \overrightarrow{P_0P}=0\]
        Så får vi planens ligning.
    \subsubsection*{Beregning}
        Vi ved at linjen \(l\) står vinkelret på en plan \(\pi\), der indeholder punktet \(C \begin{pmatrix} 3 & 4 & 8 \end{pmatrix}\).\\\\
        \(C \defeq \langle 3|4|8\rangle = \textcolor{blue}{\begin{pmatrix} 3 & 4 & 8 \end{pmatrix}}\)\\\\
        \(\overrightarrow{P_0P} \defeq \begin{pmatrix} x \\ y \\ z\end{pmatrix} - C^{\%T}= \textcolor{blue}{\begin{pmatrix} x-3 \\ y-4 \\ z-8  \end{pmatrix}}\)\\\\\\
        \(\overrightarrow{AB} \bullet \overrightarrow{P_0P}=0\)
        \[\textcolor{blue}{x+2y-4z+21=0}\]
    \subsubsection*{Konklusion}
        Planen \(\pi\)'s ligning kan skrives ved:
        \[x+2y-4z+21=0\]
\clearpage
\subsection*{4c: Bestem koordinaterne til skæringspunktet mellem linjen \textit{l} og planen \(\pi\)}
    \subsubsection*{Metode}
        Vi har allerede bestemt en parameterfremstilling for linjen \(l\) og planens ligning.\\
        Vi opstiller 4 ligninger med 4 ubekendte, hvor vi bruger planens ligning og parameterfremstillingen for \(l\).
    \subsubsection*{Beregning}
        \(eq1 \defeq x=A[1]+t\cdot \overrightarrow{AB}[1]\)\\
        \(eq2 \defeq y=A[2]+t\cdot \overrightarrow{AB}[2]\)\\
        \(eq3 \defeq z=A[3]+t\cdot \overrightarrow{AB}[3]\)\\\\
        \( eq4 \defeq \overrightarrow{AB} \bullet \overrightarrow{P_0P}=0\)\\\\
        \(solve \left(\{eq1, eq2, eq3, eq4\},\{t,x,y,z\}\right)\)
        \[\textcolor{blue}{t=-1,\, x=1,\, y=3,\, z=7}\]
    \subsubsection*{Konklusion}
    Skæringspunktet mellem linjen \(l\) og planen \(\pi\) er i punktet \(\begin{pmatrix} 1 & 3 & 7 \end{pmatrix}\).
\end{document}