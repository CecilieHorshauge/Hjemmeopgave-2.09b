\documentclass[../main.tex]{subfiles}
\begin{document}
\section*{Opgave 1}
\subsection*{1a: Bestem koordinaterne til hjørnerne \textit{C}, \textit{E}, \textit{F} og \textit{G}.}
    \subsubsection*{Metode}
        Ved at betragte figuren og udnytte at alle sider står vinkelret på hinanden, kan vi ræsonnere os frem til koordinaterne til \(C,\, D,\, F \text{ og } G\).
    \subsubsection*{Beregning}
        \(A \defeq \langle 0{\vert 0\vert}0 \rangle\)\\
        \(B \defeq \langle 6{\vert 0\vert}0 \rangle\)\\
        \(D \defeq \langle 0{\vert 10\vert}0\rangle\)\\
        \(H \defeq \langle 0{\vert 0\vert}8 \rangle\)\\\\
        \(C \defeq B + D = \textcolor{blue}{\begin{pmatrix} 6 & 10& 0\end{pmatrix}}\)\hspace*{3 cm}
        \(F \defeq B + D + H =\textcolor{blue}{\begin{pmatrix} 6& 10& 8\end{pmatrix}}\)\\\\\\
        \(E \defeq B + H = \textcolor{blue}{\begin{pmatrix} 6&10&8\end{pmatrix}}\)\hspace*{3 cm}
        \(G \defeq D + H = \textcolor{blue}{\begin{pmatrix} 0& 10& 8\end{pmatrix}}\)
    \subsubsection*{Konklusion}
        Koordinaterne til kassens øvrige hjørner er blevet bestemt til:\\\\
        \(C = \begin{pmatrix} 6 & 10 & 0\end{pmatrix}\)\hspace*{1.4 cm}
        \(F =\begin{pmatrix} 6 & 10 & 8\end{pmatrix}\)\hspace*{1.4 cm}
        \(E = \begin{pmatrix} 6 & 10 & 8\end{pmatrix}\)\hspace*{1.4 cm}
        \(G = \begin{pmatrix} 0 & 10 & 8\end{pmatrix}\)
\clearpage
\subsection*{1b: Bestem en ligning for den plan \(\alpha\), der går gennem hjørnerne \textit{B}, \textit{D} og \textit{H}. }
    \subsubsection*{Metode}
        Først bestemmes vektor \(\overrightarrow{BH}\) og \(\overrightarrow{BD}\), de udspænder da et plan \(\alpha\) der går gennem \(B,\, D,\text{ og } H\).\\
        Vi krydser \(\overrightarrow{BH}\) og \(\overrightarrow{BD}\) for at bestemme en normalvektor til planen.\\
        Dernest prikker vi normalvektoren med en vilkårlig vektor i planen.
        \[\overrightarrow{n}\bullet \left(\begin{pmatrix} x \\ y \\ z \end{pmatrix}-\begin{pmatrix}B_x \\B_y\\B_z \end{pmatrix}\right)=0\]
        Vi får planens ligning.
    \subsubsection*{Beregning}
        \(\overrightarrow{BD} \defeq D^{\%T}-B^{\%T} = \textcolor{blue}{\begin{pmatrix} -6 \\ 10 \\ 0\end{pmatrix}}\)\\\\\\
        \(\overrightarrow{BH} \defeq H^{\%T}-B^{\%T} = \textcolor{blue}{\begin{pmatrix} -6 \\ 0 \\ 8\end{pmatrix}}\)\\\\\\
        \(\overrightarrow{n} \defeq \overrightarrow{BD} \times \overrightarrow{BH} = \textcolor{blue}{\begin{pmatrix} 80 \\ 48 \\ 60\end{pmatrix}}\)\\\\\\
        \(\overrightarrow{n} \bullet \left(\begin{pmatrix} x \\ y\\ z\end{pmatrix} - B^{\%T}\right)=0\)
        \textcolor{blue}{\[80x+48y+60z-480=0\]}
    \subsubsection*{Konklusion}
        Planen \(\alpha\)'s ligning kan skrives ved: \(\; \; \; 80x+48y+60z-480=0\)
\clearpage
\subsection*{1c: Bestem koordinaterne til skæringspunktet mellem planen \(\alpha\) og linjen \(l\), der går gennem punkterne \textit{A} og \textit{F}.}
    \subsubsection*{Metode}
        Først bestemmer jeg en parameterfremstilling for linjen \(l\).\\
        Dernæst udnytter jeg at vi allerede har bestemt planens ligning. Vi opstiller 4 ligninger med 4 ubekendte, hvor vi bruger planens ligning og parameterfremstillingen for \(l\).
    \subsubsection*{Beregning}
        \hspace{\parindent} \textcolor{violet}{\textbf{\underline{Parameterfremstilling for \textit{l}}}}\\
        \par \(\overrightarrow{AF} \defeq F^{\%T}-A^{\%T} = \textcolor{blue}{\begin{pmatrix} 6 \\ 10\\ 8\end{pmatrix}}\)\\\\
        \par \(\langle x, y, z \rangle = A^{\%T}+ t\cdot \overrightarrow{AF}\)
        \par \textcolor{blue}{\[\begin{pmatrix} x \\ y\\ z\end{pmatrix}=\begin{pmatrix} 6t \\ 10t\\ 8t\end{pmatrix}\]}
        \par \textcolor{violet}{\textbf{\underline{Bestemmelse af skæringspunktet}}}\\
        \par \(lign1 \defeq 80 x-480+48y+60z=0\)\\
        \par \(lign2 \defeq A[1]+t\cdot \overrightarrow{AF}[1]\)\\
        \par \(lign3 \defeq A[2]+t\cdot \overrightarrow{AF}[2]\)\\
        \par \(lign4 \defeq A[3]+t\cdot \overrightarrow{AF}[3]\)\\
        \par \(solve(\{lign1,\, lign2,\, lign 3,\, lign4\},\{t,x,y,z\})\)
        \par \textcolor{blue}{\[t=\frac{1}{3},\; x=2,\; y=\frac{10}{3},\; z=\frac{8}{3}\]}
    \subsubsection*{Konklusion}
        Skæringspunktet mellem planen \(\alpha\) og linjen \(l\) er i punktet \(\displaystyle \begin{pmatrix}2 & \frac{10}{3} & \frac{8}{3} \end{pmatrix}\)
        \end{document}